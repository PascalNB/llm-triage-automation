\section{Planning}


In this section we will shortly discuss the planning of the study. The study has been split into six parts, as can be seen in the table below. Note that this planning is merely meant as a guideline, and is not set in stone.

More examples on how to do a planning table you can see in \url{http://www.martin-kumm.de/wiki/doku.php?id=05Misc:A_LaTeX_package_for_gantt_plots}

\begin{figure*}[t!]
	\centering
	%	\noindent\resizebox{0.6\textwidth}{!}{
	\begin{gantt}[xunitlength=0.5cm,fontsize=\small,titlefontsize=\small,drawledgerline=true]{14}{18} %(1)lines (2) columns
		
		\begin{ganttitle} %Month
			\titleelement{\textbf{Planning Table}}{18}
		\end{ganttitle}
		
		\begin{ganttitle} %Month
			\titleelement{June}{4}
			\titleelement{July}{4}
			\titleelement{August}{4}
			\titleelement{September}{4}
			\titleelement{October}{2} 
		\end{ganttitle}
		
		\begin{ganttitle} % Week number
			\numtitle{23}{1}{40}{1}
		\end{ganttitle}
		
		\ganttbar{Proposal}{0}{2}
		
		\ganttgroup{Rel. Work}{2}{5}
		\ganttbar{task 2}{2}{2}
		\ganttbarcon[pattern=crosshatch,color=blue]{task 3}{4}{1} %(1)start point; (2) number of weeks
		\ganttbarcon{task 4}{5}{2}  
%		\ganttcon{5}{5}{6}{6} %(1)vertical bar 
		
		\ganttgroup{Holidays*}{7}{3}
		
		\ganttgroup{Analysis}{10}{6}
		\ganttbar{task 5}{10}{2}
		\ganttbarcon[color=red]{task 6}{12}{2}
		\ganttbarcon{task 7}{14}{2}
		
		\ganttmilestone{Deadline}{17}
	\end{gantt}	
	%	}
%	\caption{Planning Table}
\end{figure*}


The research topics part consists solely of a literature study that focuses on ... All relevant information learned from this will be integrated in a survey that will form the first part of the thesis.

Following the research topics are each of the research questions, with time allotted at the end of each research question to integrate the results into the thesis. 