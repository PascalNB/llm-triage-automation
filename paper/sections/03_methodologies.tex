\section{Methodologies}
\label{sec:methodologies}

The following sections provide descriptions about the methodologies that will be used to answer the research questions.
It will act as a roadmap that will be followed throughout the research to ensure its transparency and reproducibility.

%-------------------------------------------------------------------------------
% RQ1
%-------------------------------------------------------------------------------

\subsection{On answering RQ1}
\label{subsec:on-answering-rq1}

The goal of the first research question is to find in which parts of the triage process LLMs can be applied, and how
they can be applied.
To answer this, the following steps will be made:

\begin{enumerate}
    \item A literature review will be conducted to identify existing and proposed ways in which the triage process can
    be (partially) automated.
    \item Examples of triaging by professional security analysts will be used to discover the process steps where LLMs
    can be applied.
    \item An implementation will be made in Python using a Jupyter notebook to automatically run tasks on different
    LLMs.
    The data used in these tasks will be provided by Northwave Cyber Security.
\end{enumerate}

Besides GPT-4, the exact choice of LLMs is still up for discussion.
The other models will be used through the Ollama\ \citep{ollama} API because of its simplicity to deploy and use openly
available models.

%-------------------------------------------------------------------------------
% RQ2
%-------------------------------------------------------------------------------

\subsection{On answering RQ2}
\label{subsec:on-answering-rq2}

The goal of the second research question is to discover suitable evaluation metrics to assess the performance of the
different LLMs.
This will be done through the following steps:

\begin{enumerate}
    \item Quantitative measures will be identified to reflect the performance of models (e.g.\ accuracy, precision,
    F-1 score).
    \item Consistent prompts are engineered to eliminate bias across different LLMs.
    These prompts are unique per type of alarm, but future scalability and other alarm types should be considered.
    \item An analysis framework is established to systematically test models.
    This will also be implemented in a Jupyter notebook for better replicability.
\end{enumerate}

%-------------------------------------------------------------------------------
% RQ3
%-------------------------------------------------------------------------------

\subsection{On answering RQ3}
\label{subsec:on-answering-rq3}
After answering RQ2, the quantitative analysis framework can be used to compare the performance of different LLMs.
Answering the third research question will be done as follows:

\begin{enumerate}
    \item Triage examples as test cases are collected.
    For the scope of this project, this will consist of three types of alarms.
    \item Each model is tested on all cases and comparisons will be made based on the scoring.
    \item Using the scores, conclusions will be drawn about the strengths and weaknesses of each model.
\end{enumerate}