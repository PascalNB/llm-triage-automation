\section{Conclusion}
\label{sec:conclusion}

This study aimed to investigate the integration of LLMs to optimize the cybersecurity triage process and offer a
comparison of various LLMs through a systematic evaluation.
Where proposed and existing optimizations show promising results, many steps in triage still require analyzing and
understanding natural language.
A literature review has shown that general LLMs are capable of proficiently executing these steps.
Although fine-tuned models have been the stat-of-the-art solution for specific tasks, recent general LLMs have shown a
comparable competence.

After identifying suitable performance metrics, comparative analysis of different LLMs revealed that GPT-4
consistently performed at a high level when completing cybersecurity announcement detection and MITRE ATT\&CK tactic
classification.
Llama 3 and Mistral showed a competitive performance on all tasks, while the 3.8B parameter version of Phi-3
achieves good results on simple tasks.

Despite limitations on the amount and type of data used, this research provides a comprehensive framework for
evaluating LLMs in cybersecurity triage.
Further research is needed to analyze the performance of LLMs in other parts of the triage process, and establish a
comparison including fine-tuned or larger models.