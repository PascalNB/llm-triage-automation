% context
% problem
% difference from literature
% research
% structure
\section{Introduction}
\label{sec:introduction}

(TODO: numbers on security incidents) % TODO

The 2023 Cost of a Data Breach Report by\ \citet{ibm2023cost} concludes that the average cost of a security breach in
2023 was 4.45M USD, marking an increase of 2.3\% since 2022 and a 15.3\% increase compared to 2020.
Considering that only 1 in 3 breaches are identified by an organization's security team, and that organizations with
high levels of incident response planning saved 1.49M USD on average, it shows the pressing need for security
investments in training, thread detection and response technologies.

One such investment is the use of Security Operation Centers (SOCs).
Organizations use SOCs to respond to security incidents in real-time.
SOCs consist of security analysts that investigate data from various sources such as Security Information and Event
Management (SIEM) systems.
The SIEM systems collect log data from a large number of sources such as network devices and applications within the
organization's system.
Based on rules, patterns and conditions, anomalies and suspicious activities are identified and alarms are created.

The number of alarms is large, ranging from hundreds to thousands per day, of which a large portion are false positives
or low priority.
The volume and complexity of the alarms causes SOCs to miss serious attacks and inadvertently contributes to mistakes in
the analysis.
Besides that, it leads to security teams experiencing fatigue, and it contributes to internal friction and
turnover\ \citep{orca2022fatigue}.

Since the SOCs cannot respond to every single alarm, identifying the severity of alarms is an important step in the
incident response workflow.
This process, triage, involves understanding the impact of an alarm, correlating it with other alarms and identifying
potential future goals of adversaries to conclude its severity.
By prioritizing alarms, SOCs can focus their resources on high-severity alarms first, thus mitigating damages and
reducing costs.

There are many proposed and implemented techniques to optimize triage and the SOC workflow.
For example, Security Orchestration, Automation, and Response (SOAR) platforms have streamlined parts of the process by
automating routine tasks, but many steps of triage still require human judgment to make adequate
decisions\ \citep{chuvakin2019triaging}.
Consequentially, the triage process is prone to human error.
This, in combination with the volume and complexity of alarms, presses the need for the automation of triage.

The field of Artificial Intelligence (AI) has the potential to significantly impact the automation of triage.
It involves the use of machines to perform tasks that mimic human actions such as reasoning, problem-solving and
learning\ \citep{oed:ai}.
Machine Learning (ML) allows systems to solve problems by analyzing patterns in data and making
predictions and decisions without explicit programming\ \citep{oed:ml}.
One subfield of ML is Natural Language Processing (NLP).
NLP involves using computational approaches to process and transcribe natural-language texts with further goals such as
translation, summarization, sentiment assessment or generation of texts.
It plays a growing role in streamlining and automating business operations, and increasing
productivity\ \citep{ibm:nlp}.

One application of NLP is that of Large Language Models (LLMs).
These models have been trained on immense amounts of natural-language data and are capable of understanding and
generating texts to perform a wide range of tasks\ \citep{ibm:llm}.
They are designed to be applied in any domain or industry, erasing the need to create or train a domain-specific ML
model.
Their ability to identify contextual relationships and recognize complex patterns\ \citep{bordt2024data} in a short
amount of time makes LLMs the perfect entrypoint to automate triage and thus optimize the incident response workflow.

This research aims to explore the potential of LLMs in optimizing the triage process, establish ways to
evaluate the performance of LLMs in cybersecurity, and present a comparison of different models when automating triage
steps.
To pursue this goal, the following research questions (RQ) provide the basis of this research:

\begin{itemize}
    \item \textbf{RQ1:} How can LLMs be integrated into the existing incident response workflow to streamline the triage
    process?
    \item \textbf{RQ2:} What suitable evaluation metrics should be used to assess the performance of LLMs in
    cybersecurity triage?
    \item \textbf{RQ3:} How do different LLMs compare in performance when optimizing the cybersecurity triage process?
\end{itemize}

(TODO: Structure) % TODO
