% context
% problem
% difference from literature
% research
% structure
\section{Introduction}
\label{sec:introduction}

(TODO: numbers on security incidents) % TODO

The 2023 Cost of a Data Breach Report by\ \citet{ibm2023cost} concludes that the average cost of a security breach in
2023 was 4.45M USD, marking an increase of 2.3\% since 2022 and a 15.3\% increase compared to 2020.
Considering that only 1 in 3 breaches are identified by an organization's security team, and that organizations with
high levels of incident response planning saved 1.49M USD on average, it shows the pressing need for security
investments in training, thread detection and response technologies.

\subsection{Triage}
\label{subsec:intro-triage}

Organizations use Security Operation Centers (SOC) to respond to security incidents in real-time.
SOCs consist of security analysts that investigate data from various sources such as Security Information and Event
Management (SIEM) systems.
The SIEM systems collect log data from a large number of sources such as network devices and applications within the
organization's system.
Based on rules, patterns and conditions, anomalies and suspicious activities are identified and alarms are created.

The number of alarms is large, ranging from hundreds to thousands per day.
This causes security teams to experience fatigue, and it contributes to internal friction and turnover.
Besides that, a large portion of the alarms are false positives or low priority\ \citep{orca2022fatigue}.

Since the SOCs cannot respond to every single alarm, identifying the severity of alarms is an important step in the
incident response workflow.
This process, triage, involves understanding the impact of an alarm, correlating it with other alarms and identifying
potential future goals of adversaries to conclude its severity.
By prioritizing alarms, SOCs can focus their resources on high-severity alarms first, thus mitigating damages and
reducing costs.

Although Security Orchestration, Automation, and Response (SOAR) platforms have streamlined parts of the process by
automating routine tasks, many steps still require human judgment to make adequate
decisions\ \citep{chuvakin2019triaging}.
Consequentially, the triage process is prone to human error.
This, in combination with the volume and complexity of alarms, presses the need for the automation of triage.

\subsection{Large Language Models}
\label{subsec:intro-definitions}

Artificial Intelligence (AI) can potentially play a big role when automating triage.
This field of study involves the use of machines to perform tasks that require human intelligence such as reasoning,
problem-solving and learning\ \citep{oed:ai}.
Machine Learning (ML) is a type of AI that allows systems to solve problems by analyzing patterns in data and making
predictions and decisions without explicit programming\ \citep{oed:ml}.
One purpose of ML is Natural Language Processing (NLP).
NLP involves using computational approaches to process natural-language texts with goals such as translation,
summarization, sentiment assessment or generation of texts.
It plays a growing role in streamlining and automating business operations, and increasing
productivity\ \citep{ibm:siem}.

A part of NLP is Natural Language Understanding (NLU) which aims to comprehend meaning and intent in natural language.
Opposite of this, Natural Language Generation (NLG) focuses on generating original human-like texts.
To achieve NLU and NLG, language models make use of so-called encoders and decoders.
The purpose of encoders is to turn input texts into fixed-size vectors that act as abstract representations.
Language models then use decoders to turn such representations into a generated target output.
This approach works well on tasks that map input sequences to output sequences (sec2sec), such as language
translation\ \citep{sutskever2014sequence, cho2014learning}.

In 2014,\ \citet{bahdanau2014neural} introduced the concept of attention which rids the encoders of creating
fixed-length vectors, allowing language models to focus on the most relevant parts of texts and enabling operations on
much longer input sequences.
Based on this,\ \citet{vaswani2017attention} developed the transformer architecture in 2017, setting the precedent for
modern LLMs.
Transformers are superior in quality, more parallelizable, thus faster, and take less time to train.

Early well-known examples of such transformer-based models are BERT
(Bidirectional Encoder Representations from Transformers)\ \citep{devlin2018bert}, and
GPT (Generative Pre-trained Transformer)\ \citep{radford2018improving}.
\begin{itemize}
    \item BERT was specifically designed as a pre-trained model to be easily fine-tuned for a wide range of tasks such
    as answering questions and natural language inference, without the need of task-specific architecture.
    In the cybersecurity domain, BERT models have been fine-tuned to detect malicious software\ \citep{rahali2021malbert}
    and phishing emails\ \citep{lee2020catbert}.
    \item GPT is pre-trained on a large amount of unlabeled data and designed to generate coherent context-specific
    text.
    Like BERT, it requires fine-tuning to adapt the model to specific tasks.
\end{itemize}

(TODO: disadvantages fine-tuned models vs general models) % TODO

\subsection{Research Structure}
\label{subsec:intro-research-structure}

% TODO citations

This research aims to explore the potential of LLMs in optimizing the triage process, as well as evaluate the
performance of different LLMs (e.g.,\ GPT-4\ \citep{achiam2023gpt}, Llama 3\ [?],
Mistral\ \citep{jiang2023mistral}) and establish a comparison of these models.
To pursue our goal, we define the following research questions (RQ) as the basis of our research:

\begin{itemize}
    \item \textbf{RQ1:} How can LLMs be integrated into the existing incident response workflow to streamline the triage process?
    \item \textbf{RQ2:} What suitable evaluation metrics should be used to assess the performance of LLMs in cybersecurity triage?
    \item \textbf{RQ3:} How do different LLMs compare in performance when optimizing the cybersecurity triage process?
\end{itemize}

(TODO: Structure) % TODO
