% context
% problem
% difference from literature
% research
% structure
\section{Introduction}
\label{sec:introduction}

Introduction and background.
SOC\@.
What is Triage.

\subsection{Problem}
\label{subsec:intro-problem}

Problem with triage.

\begin{enumerate}
    \item say that security attacks are happening very often (get numbers).
    \item say that the alarms are handled by security operation centers (SOC).
    \item the tool that centralizes alarms is the SIEM\@.
\end{enumerate}

\subsection{AI, ML, NLP and LLM}
\label{subsec:intro-definitions}

History and definitions.

\subsection{Research Questions}
\label{subsec:intro-research-questions}

%TODO citations

This research aims to explore the potential of LLMs in optimizing the triage process, as well as evaluate the
performance of different models (e.g.,\ GPT-4\ \citep{achiam2023gpt}, Llama 3, Mistral) and establish a comparison of
these models.
To pursue our goal, we define the following research questions (RQ) as the basis of our research:

\begin{itemize}
    \item \textbf{RQ1:} How can LLMs be integrated into the existing incident response workflow to streamline the triage process?
    \item \textbf{RQ2:} What suitable evaluation metrics should be used to assess the performance of LLMs in cybersecurity triage?
    \item \textbf{RQ3:} How do different LLMs compare in performance when optimizing the cybersecurity triage process?
\end{itemize}
