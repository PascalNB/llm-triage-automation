\section{Evaluation of LLMs in Cybersecurity Tasks}
\label{sec:rq2}

This section intends to answer RQ2:
\textit{What suitable evaluation metrics should be used to assess the performance of LLMs in cybersecurity triage?}
Firstly, existing evaluation metrics for LLMs are identified.
Finally, the most suitable metrics are determined to establish a testing framework for LLMs in the context of
cybersecurity triage-related tasks.

\subsection{Existing LLM Evaluation Metrics}
\label{subsec:rq2-existing-metrics}

Due to the inherent ambiguity of human language, it is challenging to evaluate the output of an LLM\@.
Outputs of LLMs are not numerical in nature, but evaluation algorithms should produce a numerical score.
This necessitates the use of sophisticated evaluation metrics.
Besides simple human evaluation techniques like expert reviews and crowdsourcing, there are some notable automated
metrics to measure LLM performance:
\begin{itemize}
    \item The BLEU\ \citep{papineni2002bleu} score is specifically designed to test machine translation by matching
    output texts with reference texts.
    \item The ROUGE\ \citep{lin2004rouge} score is used to evaluate text summaries by comparing model outputs with
    expected outputs.
\end{itemize}
These evaluation scores are purely statistical and thus reliable, but do not consider the nuances of semantics.
They demonstrate a low correlation with human judgments, particularly in tasks related to creativity and
diversity\ \citep{liu2023gpteval}.

On the other hand, NLP-based evaluation techniques are more accurate but less reliable.
Metrics such as BERTScore\ \citep{zhang2019bertscore} and BLEURT\ \citep{sellam2020bleurt} do not use LLMs, but provide
a score by comparing generated and reference texts while taking semantics into account.
\citet{liu2023gpteval} propose G-EVAL\dots

\subsection{Evaluating use LLMs in Triage}
\label{subsec:rq2-evaluating-triage}

(TODO: apply metrics to context) % TODO