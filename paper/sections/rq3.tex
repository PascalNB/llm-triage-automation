\section{Comparing LLMs in Cybersecurity Triage}
\label{sec:rq3}

This section intends to answer RQ3:
\textit{How do different LLMs compare in performance when optimizing the cybersecurity triage process?}
The automation steps in section\ \ref{subsec:rq1-llms-in-context} are combined with the evaluation metrics as described
in section\ \ref{subsec:rq2-evaluating-triage} to perform a comparison between various LLMs acros different tasks.

\subsection{Comparison Framework and Setup}
\label{subsec:rq3-comparison-framework}

The first LLM that is included in the comparison is GPT-4.
The other models selected for the comparison are openly available and accessible through the
Ollama\ \citep{ollama} library.
The Ollama language model platform is chosen for its ease of use and its simplicity to deploy the most recent
state-of-the-art models.
It is capable of running a local server that receives API calls, enabling the execution of operations on models from
external applications like a Jupyter Notebook\ \citep{kluyver2016jupyter}.

Most models are released as a collection of multiple variants, each with a different parameter count.
A model with a large number of parameters is expected to perform better, but requires more computational resources
in contrast to a small model.
Due to computational limits, only models with parameter counts of up to 14 billion have been selected.
The selected models are Aya 23\ \citep{aryabumi2024aya}, Code Llama\ \citep{roziere2023code},
Gemma\ \citep{team2024gemma}, Llama 3, Mistral\ \citep{jiang2023mistral} and Phi-3\ \citep{abdin2024phi}.
The different models and their sizes and characteristics are as follows:
\begin{itemize}
    \item \textbf{GPT-4:} Developed by OpenAI as their fourth generation in their GPT series.
    \item \textbf{Aya 23\ 8B:} A multilingual model developed by Cohere that supports 23 languages.
    \item \textbf{Code Llama 13B:} A fine-tuned Llama model by Meta and designed for generating and discussing code.
    \item \textbf{Gemma 7B:} Part of a family of lightweight models developed by Google DeepMind.
    \item \textbf{Gemma 2B:} The smallest model variant of the Gemma model family.
    \item \textbf{Llama 3\ 8B:} A capable pre-trained model developed by Meta and the third in their Llama series.
    \item \textbf{Mistral 7B:} A versatile model developed by Mistral AI\@.
    \item \textbf{Phi-3 14B:} A lightweight open model developed by Microsoft tailored to logic reasoning.
    \item \textbf{Phi-3 3.8B:} The smallest model variant in the Phi-3 series.
\end{itemize}

\subsubsection{Announcement Detection}
% TODO

\subsection{Comparison Results}
\label{subsec:rq3-comparison-results}
% TODO

\subsubsection{Announcement Detection}

See table\ \ref{tab:announcement-detection}

Notably, Phi-3 Medium is the only model that struggles to adhere to the requested output format.
The error rates are 0.03, 0.15 and 0.75 for long, medium and short prompts respectively.

\begin{table*}
    \caption{%
        F1-scores, median evaluation times in seconds and error rates of each model for different prompt sizes when
        performing announcement detection.
    }
    \label{tab:announcement-detection}
    \resizebox{\linewidth}{!}{
        \begin{tabular}{lccccccccc}
            \toprule
            \multicolumn{1}{c}{\multirow{2}{*}{\textbf{Model}}} & \multicolumn{3}{c}{\textsc{Long prompt}} & \multicolumn{3}{c}{\textsc{Medium prompt}} & \multicolumn{3}{c}{\textsc{Short prompt}} \\
            \cmidrule(lr){2-4} \cmidrule(lr){5-7} \cmidrule(lr){8-10}
            & F1-score & Median time    & Error rate & F1-score & Median time    & Error rate & F1-score & Median time   & Error rate \\
            \midrule
            \textsc{GPT-4}          & 1.000    & \textit{8.456} & 0.000      & 0.950    & \textit{7.541} & 0.000      & 0.621    & \textit{10.940}      & 0.000      \\
            \textsc{Aya 23\ 8B}     & 0.844    & 1.513          & 0.000      & 0.809    & 1.421          & 0.000      & 0.516    & 1.394           & 0.000      \\
            \textsc{Code Llama 13B} & 1.000    & 3.426          & 0.000      & 0.706    & 3.389          & 0.000      & 0.529    & 3.315           & 0.000      \\
            \textsc{Gemma 7B}       & 0.710    & 1.787          & 0.000      & 0.710    & 1.687          & 0.000      & 0.684    & 1.680           & 0.000      \\
            \textsc{Gemma 2B}       & 0.788    & 0.194          & 0.000      & 0.333    & 0.192          & 0.000      & 0.667    & 0.192           & 0.000      \\
            \textsc{Llama 3\ 8B}    & 1.000    & 0.656          & 0.000      & 0.900    & 0.619          & 0.000      & 0.837    & 0.623           & 0.000      \\
            \textsc{Mistral 7B}     & 1.000    & 0.487          & 0.000      & 0.895    & 0.411          & 0.000      & 0.743    & 0.405           & 0.000      \\
            \textsc{Phi-3 14B}      & 0.842    & 3.946          & 0.025      & 0.452    & 2.952          & 0.150      & 0.600    & 2.974           & 0.750      \\
            \textsc{Phi-3 3.8B}     & 1.000    & 0.355          & 0.000      & 0.927    & 0.375          & 0.000      & 0.706    & 0.875           & 0.000      \\
            \bottomrule
        \end{tabular}
    }
\end{table*}

\subsubsection{Announcement Tactic Detection}

See table\ \ref{tab:tactic-detection}

\begin{table*}
    \caption{Accuracies, median evaluation times in seconds and error rates of each model for different prompt sizes
    when identifying MITRE ATT\&CK tactics of potential alarms following an announcement.}
    \label{tab:tactic-detection}
    \begin{tabular}{lcccccc}
        \toprule
        \multicolumn{1}{c}{\multirow{2}{*}{\textbf{Model}}} & \multicolumn{3}{c}{\textsc{Long prompt}} & \multicolumn{3}{c}{\textsc{Medium prompt}} \\
        \cmidrule(lr){2-4} \cmidrule(lr){5-7}
        & Accuracy       & Median time    & Error rate & Accuracy       & Median time    & Error rate \\
        \midrule
        \textsc{GPT-4}          & \textbf{0.800} & \textit{2.586} & 0.000      & \textbf{0.800} & \textit{2.539}       & 0.000      \\
        \textsc{Aya 23\ 8B}     & 0.450          & 1.794          & 0.000      & 0.250          & 2.292          & 0.000      \\
        \textsc{Code Llama 13B} & 0.450          & 3.658          & 0.000      & 0.600          & 3.989          & 0.000      \\
        \textsc{Gemma 7B}       & 0.650          & 2.260          & 0.000      & 0.167          & 4.214          & 0.700      \\
        \textsc{Gemma 2B}       & 0.150          & 0.212          & 0.000      & 0.250          & 0.213          & 0.000      \\
        \textsc{Llama 3\ 8B}    & 0.550          & 0.680          & 0.000      & 0.450          & 0.817          & 0.000      \\
        \textsc{Mistral 7B}     & 0.650          & 0.479          & 0.000      & 0.600          & 0.481          & 0.000      \\
        \textsc{Phi-3 14B}      & 0.700          & 3.885          & 0.000      & 0.200          & 6.031          & 0.000      \\
        \textsc{Phi-3 3.8B}     & 0.300          & 0.882          & 0.000      & 0.350          & 0.894          & 0.000      \\
        \bottomrule
    \end{tabular}
\end{table*}