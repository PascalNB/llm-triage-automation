% context
% problem
% difference from literature
% proposal
% structure
\section{Introduction}
\label{sec:introduction}

The growing landscape of cyber threats has led to the introduction of Security Information and Even Management (SIEM)
systems, which help organizations to detect and respond to security incidents.
The systems monitor and analyze large amounts of data from various sources, and create alarms when certain conditions
are met.\ \citep{ibmsiem}
Security analysts have a task to filter out false positives and prioritize true events.
This process, triaging, involves understanding the impact of an alarm, correlating it with other alarms and identifying
potential future goals of adversaries to conclude the severity of the alarm.
In practice, only a small portion of alarms are thoroughly investigated since the volume and complexity of alarms make
the task slow and difficult.\ \citep{chen2024elevating}
Besides this, the trivial approaches are prone to human error, which presses the need for automation.

Although Security Orchestration, Automation, and Response (SOAR) platforms have streamlined parts of the process by
automating routine tasks, many steps still require human judgment to make adequate decisions.
\ \citep{chuvakin2019triaging}
The recent development of Large Language Models (LLMs) can introduce new ways to optimize the triage process.
By utilizing their ability to quickly process vast amounts of unstructured text using Natural Language Processing (NLP),
LLMs are well-suited to extract relevant information.\ \citep{chaudhary2024ai}
Integrating the use of LLMs into key parts of the triage process can speed up alarm classification and allow for timely
responses.
The MITRE ATT\&CK knowledge base consists of adversary tactics, techniques and examples based on real-world
observations\ \citep{strom2018mitre} and can thus be used to identify where in the kill chain the alarm is located.
This knowledge base can act as a basis for the use of LLMs to automate parts of the alarm response workflow.

This research aims to explore the potential of LLMs in optimizing the triage process, as well as evaluate the
performance of different models (e.g.,\ GPT-4\ \citep{achiam2023gpt}, llama3, mistral) and establish a comparison of
these models.
To pursue our goal, we define the following research questions (RQ) as the basis of our research:

\begin{itemize}
    \item \textbf{RQ1:} How can LLMs be integrated into the existing incident response workflow to streamline the triage process?
    \item \textbf{RQ2:} How can suitable evaluation metrics be used to assess the performance of LLMs in cybersecurity triage?
    \item \textbf{RQ3:} How do different LLMs compare in performance when categorizing cybersecurity incidents as high, medium, low or no risk?
\end{itemize}

The remainder of this proposal is organized as follows:
Firstly, section\ \ref{sec:related-work} will discuss literature about solutions in cybersecurity that are relevant to
this study.
Secondly, section\ \ref{sec:methodologies} proposed the methodologies that will be used to answer each research
question.
Besides that, section\ \ref{sec:expected-results} describes what the expected results and outcomes are.
After that, section\ \ref{sec:planning} presents a preliminary planning that will act as a guide throughout the study.
Lastly, section\ \ref{sec:use-of-ai} states how generative AI tools will be used during the study.
