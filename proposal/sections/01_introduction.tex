% context
% problem
% difference from literature
% proposal
% structure
\section{Introduction}
\label{sec:introduction}

The growing landscape of cyber threats has led to the introduction of Security Information and Even Management (SIEM)
systems, which help organizations to detect and respond to security incidents.
The systems monitor and analyze large amounts of data from various sources, and create alerts when certain conditions
are met. \citep{ibmsiem}
Security analysts have a task to filter out false positives and prioritize true events, but the volume and complexity
of alerts makes this a slow and difficult task.
The triage process involves understanding the impact of an alert, correlating it with other alerts and identifying
potential future goals of adversaries to conclude the severity of the alert.
Although Security Orchestration, Automation, and Response (SOAR) platforms have streamlined parts of the process by
automating routine tasks, many steps still require human judgement to make adequate decisions.
\citep{chuvakin2019triaging}

% TODO: difference from literature

The recent development of Large Language Models (LLMs) can introduce new ways to optimize the triage process.
By utilizing their ability to quickly process vast amounts of unstructured text, LLMs are well-suited to extract
relevant information. \citep{chaudhary2024ai}
Integrating the use of LLMs into key parts of the triage process can speed up alert classification and allow for timely
responses.

This research aims to explore the potential of LLMs in optimizing the triage process, as well as evaluate the
performance of different models (e.g.\ ChatGPT, llama3, mistral) and establish a comparison of these models.
To pursue our goal, we define the following research questions (RQ) as the basis of our research:

\begin{itemize}
    \item \textbf{RQ1:} How can LLMs be integrated into the existing incident response workflow to streamline the triage process?
    \item \textbf{RQ2:} What evaluation metrics are suitable to assess the performance of LLMs in cybersecurity triage?
    \item \textbf{RQ3:} How do different LLMs compare in performance when categorizing cybersecurity incidents as high, medium, low or no risk?
\end{itemize}

The remainder of this proposal is organized as follows:
Section 2 will discuss the approaches expected for answering each research question.
After that, we present a preliminary planning for the research questions in Section 3.
Finally, we conclude with a proposal and planning for the thesis structure in Section 4.
