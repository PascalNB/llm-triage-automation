\section{Related Work}
\label{sec:related-work}

\subsection{Optimizing Triage}
\label{subsec:optimizing-triage}

Existing and proposed solutions to improve the feasibility of triage range from trivial implementations to advanced
algorithms.
Solutions like reducing the overall volume of alarms can be achieved by fusion of alarms and increasing holism to
establish patterns of threats in networks.
Additionally, proposed graph-based solutions attempt to track the threat paths on computers, however, they only work
in specific cases and do not consider large numbers of concurrent alarm paths.\ \citep{ficke2022reconstructing}

A proposed algorithm by \citet{taheri2020cyberattack} incorporates incremental clustering to categorize cyberattacks
and find outliers to purge false alarms.
Testing the performance on data sets of intrusion detection systems resulted in outlier detection with high accuracy.

\subsection{Use of Natural Language Processing}
\label{subsec:use-of-natural-language-processing}

\citet{singh2022cyber} have used NLP to detect software vulnerabilities by using source code represented as texts,
showing the use of NLP in cybersecurity.

Since most general LLMs are not domain-specific and were not trained to use cybersecurity-related technical terms,
\citet{bayer2024cysecbert} have created a dataset and language model specifically made for the cybersecurity domain.
The model, CySecBERT, is based on BERT, a language representation model that allows fine-tuning for a wide range of
tasks.\ \citep{devlin2018bert}
CySecBERT is designed to prevent catastrophic forgetting, a phenomenon that occurs when existing models are trained on
new data and as such forget original knowledge.
The result outperforms both BERT and CyBERT (another model fine-tuned on cyber thread intelligence data\ \citep{ranade2021cybert})
in cybersecurity-related tasks.
By using domain-adaptive pre-training (DAPT)\ \citep{gururangan2020don}, more specific models within the cybersecurity
domain have been created: e.g.\ MalBERT to detect malicious software\ \citep{rahali2021malbert}, and CatBERT to detect
phishing emails by identifying social engineering techniques\ \citep{lee2020catbert}.

These examples show how LLMs can play a big role in automating cybersecurity-related tasks, however, they do not show
application in the triage process.