\section{Introduction} 
\label{sec:introduction}


\comment{The Introduction section has more or less the same structure as your abstract. The difference is that in the abstract each part is one statement/phrase, while in the introduction each part is a paragraph. So, (i) context, (ii) problem, (iii) proposal, and your most astonishing (iv) finding. Of course in the Introduction section you can give far more details than in the abstract. Avoid to copy and paste statements, re-write with different words.}

\comment{In addition to the structure that you already know you should include your \textit{research questions} between the ``proposal'' paragraph and the ``findings''. The statement that precede the RQ is something like the following: }

To pursue our goal, we have defined the following research questions (RQ) as the basis of our research: 
\begin{itemize}	
	\item \textbf{RQ1:} What are ..?
	\item \textbf{RQ2:} How to ... ?
	 \item \textbf{RQ3:} How to ...?
\end{itemize}



\comment{Please, avoid "yes or no" questions. Make questions that your reader are not able to answer immediately. Usually the questions depend on each other, it means that to answer one question you must answer the one before.}

\comment{Before a little bit of your most astonishing findings you must to introduce the structure of your paper (or proposal). Usually the text looks like the following.}
 
``The remainder of this paper (or proposal) is organized as follows. Section 2 will discuss the approaches expected for answering each research question. After that, we present a preliminary planning for the research questions in Section 3. Finally, we conclude with a proposal and planning for the thesis structure in Section 4.'' 





